%% LyX 2.3.2 created this file.  For more info, see http://www.lyx.org/.
%% Do not edit unless you really know what you are doing.
\documentclass[UTF8]{ctexrep}
\usepackage[T1]{fontenc}
\usepackage[paperwidth=52mm,paperheight=52mm]{geometry}
\geometry{verbose,tmargin=0cm,bmargin=0cm,lmargin=0cm,rmargin=0cm,headheight=0cm,headsep=0cm,footskip=0cm}
\pagestyle{empty}
\setcounter{secnumdepth}{3}
\setcounter{tocdepth}{3}

\makeatletter
\@ifundefined{date}{}{\date{}}
%%%%%%%%%%%%%%%%%%%%%%%%%%%%%% User specified LaTeX commands.
% 如果没有这一句命令,XeTeX会出错,原因参见
% http://bbs.ctex.org/viewthread.php?tid=60547
\DeclareRobustCommand\nobreakspace{\leavevmode\nobreak\ }
\usepackage{ctex}
\usepackage{xcolor}
\usepackage{fontspec}
\usepackage{fancybox} %各种样式的盒子
%
%\usepackage[a4paper,twoside,nohead,left=28mm,right=26mm,top=37mm,bottom=35mm,footskip=9.46mm]{geometry}
%\usepackage[paperwidth=96mm,paperheight=96mm,nohead,certering,margin=10mm]{geometry}
\usepackage{qrcode} %二维码宏包

%
%\usepackage{float}
%\usepackage{graphics}
%%
%
%% 103. 指定:[英文/数字]的默认用字

\setmainfont{Noto Serif}
%\setmainfont{JetBrains Mono}
%\setsansfont{HarmonyOS Sans}
%\setmonofont{Noto Mono}
%\setmonofont{JetBrains Mono}

%
%\setCJKmainfont [ ItalicFont = {JiangChengXieSong-300W} ] {Noto Serif CJK SC}
%\setCJKsansfont [ ItalicFont = {JiangChengXieHei-700W} ] {Noto Sans CJK SC}
%\setmainfont [ ItalicFont = {JiangChengXieSong-300W} ] {Noto Serif CJK SC}
%\setsansfont [ ItalicFont = {JiangChengXieHei-700W} ] {Noto Sans CJK SC}
%

\makeatother

\begin{document}
% 2021-12-29, Careone
\fontspec{Noto Serif}
\zihao{0}

%% DEBIAN_RED='#992233' #debian 红
%% UOS_BLUE='#2342ed' #UOS 蓝
%% MINT_GREEN='#aaee77' 
%% UBUNTU_YELLOW='#ee4400'
%
%% \usepackage{xcolor}
\definecolor{debianRed}{HTML}{992233} %自定义颜色1: Debian 红
\definecolor{ubuntuYellow}{HTML}{ee4400} %自定义颜色2: Ubuntu 黄
\definecolor{mintGreen}{HTML}{aaee77} %自定义颜色3: LinuxMint 绿
\definecolor{deepinBlue}{HTML}{2342ed} %自定义颜色4: Deepin 深度蓝
\definecolor{uosBlue}{HTML}{2342ed} %自定义颜色5: UOS 蓝
%%%%%%%%%%%%%%%%%%
%% P1: AtzLinux
\mbox{}
\\
%\vfil
\vspace{-0.2ex}
%
\noindent
\hfil\color{ubuntuYellow}A\color{mintGreen}t\color{deepinBlue}z~\hfil
\\
\centerline{\raisebox{1ex}{\color{debianRed}Linux}}
%
\vfill
\mbox{}
%%%%%%%%%%%%%%%%%%
%% P2: AtzLinux (大型小写字母)
%% 提示:此处使用 \sc “大型小写”命令,把所有小写字母,显示为“小号的大写英文字母格式”
\mbox{}
\\
%\vfil
\vspace{-0.2ex}
%
\sc{
\noindent
\hfil\color{ubuntuYellow}A\color{mintGreen}t\color{deepinBlue}z~\hfil
\\
\centerline{\raisebox{1ex}{\color{debianRed}Linux}}
}
%
\vfill
\mbox{}
%%%%%%%%%%%%%%
%% P3:
\newpage
\color{debianRed}
\vspace*{.01ex}\noindent\,
\qrcode{AtzLinux}\quad
%------------

\end{document}
